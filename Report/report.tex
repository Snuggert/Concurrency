\documentclass[10pt]{article}
\usepackage[margin=3cm]{geometry}
\usepackage{amssymb}
\usepackage{verbatim}
\usepackage{graphicx}
\usepackage{amsmath}
\title{\bfseries\Huge Bram van den Akker}

\date{}
\begin{document}
\title{Convergence between high performance and embedded systems}
\author{Abe Wiersma, Bram van den Akker}
\date{\today}
\maketitle
\newpage

\section{Introduction}
In this short literature study we will be taking a look at both high performance and embedded system, their difference, their growth and the convergence between the two of them. 

\subsection{Embedded systems}
For a computer system to be specified as an embedded system it has to have a dedicated function. Example embedded systems range from mp3 players, smart phones, anti-lock braking systems in a car, MRI and a rocket guidance computer (The first recognizable embedded system actually was the Apollo Guidance Computer). 

\begin{figure}[h]
  \centering
    \includegraphics[width=\textwidth]{apollo.jpg}
  \caption {Apollo Guidance Computer by Charles Stark Draper.}
\end{figure}

Embedded systems exist in very different forms. An embedded system can exist without an user interface (anti break system) or have a complex full user interface (smart phones). Opposing the desktop market were computers are limited to a few CPU architectures an embedded system can use many different CPU architectures, using the architecture optimized for the purpose. To be able to perform certain tasks peripherals like sensors, I/O ports and ethernet can be added to the system. 

\subsection{High performance systems}
Higher performance systems are powerful computer systems which can be used in a wide series of task. Common high performance computer systems are servers, desktop PC's and supercomputer.


\end{document}